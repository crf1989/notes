\documentclass{book}

\usepackage[top=1in,bottom=1in,left=1in,right=1in]{geometry}
\usepackage[colorlinks=true,linkcolor=blue]{hyperref}
\usepackage{amsmath}
\usepackage{amssymb}
\usepackage{amsthm}
\usepackage{enumerate}

\newcommand{\bra}[1]{{\langle#1|}}
\newcommand{\ket}[1]{{|#1\rangle}}
\newcommand{\inprod}[2]{\langle#1|#2\rangle}
\newcommand{\average}[1]{\langle#1\rangle}
\renewcommand{\Re}{{\rm Re}}
\renewcommand{\Im}{{\rm Im}}
\newcommand{\Tr}{{\rm Tr}}
\numberwithin{equation}{section}

\title{Notes}
\date{}
\begin{document}
%%\maketitle
\tableofcontents

\chapter{Useful formulas}
\section{Guassian Integrals}
For a positive number $a$,
\begin{equation}
  \int_{-\infty}^\infty dx e^{-ax^2}=\sqrt{\frac{\pi}{a}},\quad
  \int\frac{dz^*dz}{2\pi i}e^{-z^*az}=\frac{1}{a}\quad.
\end{equation}

For real multi-dimensional integrals,
\begin{equation}
  \int\frac{dx_1\cdots dx_n}{(2\pi)^{\frac{n}{2}}}
  e^{-\frac{1}{2}\sum_{ij}x_iA_{ij}x_j+\sum_ix_iJ_i}=
  [\det A]^{-\frac{1}{2}}e^{\frac{1}{2}\sum_{ij}J_iA^{-1}_{ij}J_j}\quad.
\end{equation}

For complex multi-dimensional integrals,
\begin{equation}
  \int\left(\prod_{i=1}^n\frac{dz_i^*dz_i}{2\pi i}\right)
  e^{-\sum_{ij}z_i^*H_{ij}z_j+\sum_i(J^*_iz_i+z_i^*J_i)}=
  [\det H]^{-1}e^{\sum_{ij}J_i^*H_{ij}J_j}\quad.
\end{equation}

For Grassmann variables integrals,
\begin{equation}
  \int\left(\prod_{i=1}^n d\eta_i^*d\eta_i\right)
  e^{-\sum_{ij}\eta_i^*H_{ij}\eta_j+\sum_i(\xi_i^*\eta_i+\eta_i^*\xi_i)}
  =[\det H]e^{\sum_{ij}\xi_i^*H_{ij}\xi_j}\quad.
\end{equation}
%%------------------------------------------------------------------------

\section{Fourier Transform of Delta Function}
The $\delta$ function can be expressed as
\begin{equation}
  \delta(x-\alpha)=\frac{1}{2\pi}\int_{-\infty}^\infty e^{ip(x-\alpha)}dp.
\end{equation}
\section{Euler Integral}
\subsection{Euler Integral of The First Kind: Beta Function}
Euler integral of the first kind: the Beta function:
\begin{equation}
  B(a,b)=\int_0^1x^{a-1}(1-x)^{b-1}dx.
\end{equation}
The Beta function has the following properties:
\begin{enumerate}[(i)]
\item Substitute $x$ with $x=1-t$ and it is easy to get
  \begin{equation}
    B(a,b)=B(b,a).
  \end{equation}

\item When $b>1$, integrate by parts (note that $x^a=x^{a-1}-x^{a-1}(1-x)$)
  \begin{equation}
    \begin{array}{rcl}
      B(a,b) &=& \displaystyle\int_0^1(1-x)^{b-1}d\frac{x^a}{a}\\\vbox to 20pt{}
      &=& \displaystyle\left.\frac{x^a(1-x)^{b-1}}{a}\right|_0^1
      +\frac{b-1}{a}\int_0^1x^a(1-x)^{b-2}dx\\\vbox to 20pt{}
      &=& \displaystyle\frac{b-1}{a}\int_0^1x^{a-1}(1-x)^{b-2}dx-
      \frac{b-1}{a}\int_0^1x^{a-1}(1-x)^{b-1}dx\\\vbox to 20pt{}
      &=&\displaystyle\frac{b-1}{a}B(a,b-1)-\frac{b-1}{a}B(a,b),
    \end{array}
  \end{equation}
  thus
  \begin{equation}
    B(a,b)=\frac{b-1}{a+b-1}B(a,b-1).
  \end{equation}
  For $a>1$, it is similar that
  \begin{equation}
    B(a,b)=\frac{a-1}{a+b-1}B(a-1,b).
  \end{equation}

  Let $n$ be a positive integer, 
  \begin{equation}
    B(n,a)=B(a,n)=\frac{1\cdot2\cdot3\cdots(n-1)}{a\cdot(a+1)\cdot(a+2)\cdots
      (a+n-1)}.
  \end{equation}
  Let $m,n$ be two positive integers,
  \begin{equation}
    B(m,n)=\frac{(n-1)!(m-1)!}{(m+n-1)!}.
  \end{equation}

\item Substitute $x$ with $x=\frac{y}{1+y}$, here $y$ is a new
  variable runs from $0$ to $\infty$, then
  \begin{equation}
    B(a,b)=\int_0^\infty\frac{y^{a-1}}{(1+y)^{a+b}}dy.
  \end{equation}

\item If $b=1-a$ and $0<a<1$ then
  \begin{equation}
    B(a,1-a)=\int_0^\infty\frac{y^{a-1}}{1+y}dy,
  \end{equation}
  this is also a Euler integral,
  \begin{equation}
    B(a,1-a)=\frac{\pi}{\sin a\pi}\quad(0<a<1),
  \end{equation}
  especially we have
  \begin{equation}
    B(\frac{1}{2},\frac{1}{2})=\pi.
  \end{equation}
\end{enumerate}
%%-----------------------------------------------------------------------------

\subsection{Euler Integral of The Second Kind: Gamma Function}
Euler integral of the second kind: the Gamma function is defined as
\begin{equation}
  \Gamma(a)=\int_0^\infty x^{a-1}e^{-x}dx.
\end{equation}

The Euler-Gauss formula:
\begin{equation}
  \Gamma(a)=\lim_{n\to\infty}n^a\frac{1\cdot2\cdot3\cdots(n-1)}
  {a\cdot(a+1)\cdot(a+2)\cdots(a+n-1)}.
\end{equation}

The Gamma Function has the following properties:
\begin{enumerate}[(i)]
\item For $a>0$, $\Gamma(a)$ is smooth.
\item Integrate by parts we shall get
  \begin{equation}
    \Gamma(a+1)=a\Gamma(a),
  \end{equation}
  repeat this formula
  \begin{equation}
    \Gamma(a+n)=(a+n-1)(a+n-1)\cdots(a+1)\Gamma(a).
  \end{equation}
  Let $n$ be a positive integer, then
  \begin{equation}
    \Gamma(n+1)=n!\quad.
  \end{equation}
\item If $a\to+0$ then
  \begin{equation}
    \Gamma(a)=\frac{\Gamma(a+1)}{a}\to+\infty.
  \end{equation}
  If $a>n+1$ the
  \begin{equation}
    \Gamma(a)>n!\quad.
  \end{equation}
\item Relation to Beta function:
  \begin{equation}
    B(a,b)=\frac{\Gamma(a)c\cdot\Gamma(b)}{\Gamma(a+b)}.
  \end{equation}
\item if $0<a<1$ then
  \begin{equation}
    \Gamma(a)\Gamma(1-a)=\frac{\pi}{\sin a\pi},
  \end{equation}
  and 
  \begin{equation}
    \Gamma(\frac{1}{2})=\sqrt{\pi}.
  \end{equation}
\item 
  \begin{equation}
    \prod_{\nu=1}^{n-1}\Gamma(\frac{\nu}{n})=
    \frac{(2\pi)^{\frac{n-1}{2}}}{\sqrt{n}}.
  \end{equation}
\item Raabe's formula:
  \begin{equation}
    \int_a^{a+1}\ln\Gamma(t)dt=\frac{1}{2}\ln2\pi+a\ln a-a,\quad a>0,
  \end{equation}
  in particular, if $a=0$ then
  \begin{equation}
    \int_0^1\ln\Gamma(t)dt=\frac{1}{2}\ln2\pi.
  \end{equation}
\item Legendre formula:
  \begin{equation}
    \Gamma(a)\Gamma(a+\frac{1}{2})=\frac{\sqrt{\pi}}{2^{2a-1}}\Gamma(2a).
  \end{equation}
\end{enumerate}
%%---------------------------------------------------------------------------

\section{Baker-Campbell-Hausdorff Formula}
Baker-Campbell-Hausdorff Formula is
\begin{equation}
  e^ABe^{-A}=\sum_{n=0}^\infty\frac{1}{n!}[A,B]_n=
  B+[A,B]+\frac{1}{2}[A,[A,B]]+\frac{1}{6}[A,[A,[A,B]]]+\cdots\quad,
\end{equation}
this formula can be proved by defining $B(\tau)=e^{\tau A}Be^{-\tau A}$
and formally integrating its equation of motion $dB/d\tau=[A,B(\tau)]$.
%%------------------------------------------------------------------------

\section{Feynman Result}
The Feynman result reads
\begin{equation}
  e^{A+B}=e^Ae^Be^{-\frac{1}{2}[A,B]},
\end{equation}
which is true only if $[A,B]$ commutes with both $A$ and $B$.

To prove it, recall that
\begin{equation}
e^{\tau(A+B)}=e^{\tau A}T_\tau\exp\left[
  \int_0^\tau d\tau'e^{-\tau'A}Be^{\tau'A}\right]
\end{equation}
and evaluate the integral for $\tau=1$.
%%---------------------------------------------------------------------------

\section{Laguerre Polynomials}
The Laguerre polynomials are solution of Laguerre's equation:
\begin{equation}
  xy''+(1-x)y'+ny=0,
\end{equation}
where $n$ is non-negative integer. The Laguerre polynomials is
\begin{equation}
  L_n(x)=\frac{e^x}{n!}\frac{d^n}{dx^n}(e^{-x}x^n)=
  \sum_{k=0}^n\frac{(-x)^k}{k!}\frac{n!}{k!(n-k)!}.
\end{equation}
The generating function is
\begin{equation}
  \frac{e^{-xt/(1-t)}}{1-t}=\sum_{n=0}^\infty L_n(x)t^n\quad.
\end{equation}
%%-------------------------------------------------------------------------

\section{Cramer's Rule}
Consider a system of $n$ linear equations of $n$ unknowns, represented in 
matrix multiplication form:
\begin{equation}
  Ax=b,
\end{equation}
where the $n\times n$ matrix $A$ has a nonzero determinant, and the
vector $x=(x_1,\cdots,x_n)^T$ is the column vector of the variables.
Then Cramer's rule states that the system has a unique solution, whose
individual values are given by:
\begin{equation}
  x_i=\frac{\det A_i}{\det A},
\end{equation}
where $A_i$ is the matrix formed by replacing the $i$-th column of $A$
by the column vector $b$.

%%-------------------------------------------------------------------------


\section{Simple Impurity Model at Zero Temperature}
The Hamiltonian of simple impurity model is defined as
\begin{equation}
  H=\sum_k\varepsilon_kc_k^\dag c_k+\sum_kV_k(c_k^\dag d+d^\dag c_k)
  +\varepsilon_0d^\dag d,
\end{equation}
let $H=H_0+V$, where
\begin{equation}
  H_0=\sum_k\varepsilon_kc_k^\dag c_k+\varepsilon_0d^\dag d,\quad
  V=\sum_kV_k(c_k^\dag d+d^\dag c_k).
\end{equation}

The Green's function is
\begin{equation}
  G(t)=-i\bra{0}Td(t)d^\dag\ket{0}=-i\bra{0}d(t)d^\dag\ket{0},
\end{equation}
apply Fourier transform on it, then
\begin{equation}
  G(\omega)=\bra{0}d\frac{1}{\omega+i0-H}d^\dag\ket{0}.
\end{equation}

Notice that
\begin{equation}
  \begin{array}{rcl}
  \displaystyle\frac{1}{\omega-H}&=&\displaystyle\frac{1}{\omega-H_0}+
  \frac{1}{\omega-H_0}V\frac{1}{\omega-H}\\\vbox to 20pt{}
  &=&\displaystyle\frac{1}{\omega-H_0}+
  \frac{1}{\omega-H_0}V\frac{1}{\omega-H_0}+
  \frac{1}{\omega-H_0}V\frac{1}{\omega-H_0}V\frac{1}{\omega-H},
  \end{array}
\end{equation}
the second term produce just 0, thus 
\begin{equation}
  \begin{array}{rcl}
  G(\omega)&=&\displaystyle\bra{0}d\frac{1}{\omega-H_0}d^\dag\ket{0}+
  \bra{0}d\frac{1}{\omega-H_0}V\frac{1}{\omega-H_0}V
  \frac{1}{\omega-H}d^\dag\ket{0}\\\vbox to 20pt{}
  &=&\displaystyle\frac{1}{\omega-\varepsilon_0}+
  \frac{1}{\omega-\varepsilon_0}\bra{0}dV\frac{1}{\omega-H_0}V
  \frac{1}{\omega-H}d^\dag\ket{0}\\\vbox to 20pt{}
  &=&\displaystyle\frac{1}{\omega-\varepsilon_0}+
  \frac{1}{\omega-\varepsilon_0}
  \bra{0}d\sum_kd^\dag c_k\frac{V_k^2}{\omega-H_0}c_k^\dag d
  \frac{1}{\omega-H}d^\dag\ket{0}\\
  &=&\displaystyle\frac{1}{\omega-\varepsilon_0}+
  \frac{1}{\omega-\varepsilon_0}
  \sum_k\frac{V_k^2}{\omega-\varepsilon_k}G(\omega).
  \end{array}
\end{equation}
Therefore 
\begin{equation}
  G^{-1}(\omega)=\omega-\varepsilon_0-
  \sum_k\frac{V_k^2}{\omega-\varepsilon_k},
\end{equation}
it can be written as
\begin{equation}
  G^{-1}(\omega)=\omega-\varepsilon_0-\int_{-\infty}^\infty d\varepsilon
  \frac{\Delta(\varepsilon)}{\omega-\varepsilon},
\end{equation}
where
\begin{equation}
  \Delta(\varepsilon)=\sum_kV_k^2\delta(\varepsilon-\varepsilon_k).
\end{equation}

Now consider $V$ is in site representation:
\begin{equation}
  V=\sum_{i}(t_{io}c_i^\dag d+t_{oi}d^\dag c_i),
\end{equation}
then we have that
\begin{equation}
  \begin{array}{rcl}
    G(\omega) &=& \displaystyle\frac{1}{\omega-\varepsilon_0}
    +\frac{1}{\omega-\varepsilon_0}\sum_{ij}t_{oi}t_{jo}\bra{0}dd^\dag c_i
    \frac{1}{\omega-H_0}c_j^\dag
    d\frac{1}{\omega-H}d^\dag\ket{0}\\\vbox to 20pt{}
    &=&\displaystyle\frac{1}{\omega-\varepsilon_0}
    +\frac{1}{\omega-\varepsilon_0}\sum_{ij}t_{oi}t_{jo}
    G_{ij}^{(o)}(\omega)G(\omega),\\
  \end{array}
\end{equation}
thus
\begin{equation}
  G^{-1}(\omega)=\omega-\varepsilon_0-\sum_{ij}t_{oi}t_{jo}G^{(o)}_{ij}(\omega),
\end{equation}
where $G^{(o)}_{ij}$ is the Green's function with one site removed.
%%----------------------------------------------------------------------------

\section{Green's Function for Simple Cubic Lattice}
The first Brillouin zone for the simple cubic lattice is the cube
\begin{equation}
  -\pi/a\le k_x<\pi/a,\quad-\pi/a\le k_x<\pi/a, \quad-\pi/a\le k_x<\pi/a,
\end{equation}
where $a$ is the lattice constant. The diagonal matrix element of
Green's function is
\begin{equation}
  G(\omega)=\frac{a^3}{(2\pi)^3}\int_{-\pi/a}^{\pi/a}dk_x
  \int_{-\pi/a}^{\pi/a}dk_y\int_{-\pi/a}^{\pi/a}dk_z
  \frac{1}{\omega-2t(\cos k_xa+\cos k_ya+\cos k_za)},
\end{equation}
introducing the variable $x=k_xa,y=k_ya,z=k_za$ we obtain
\begin{equation}
  G(\omega)=\frac{1}{(2\pi)^3}\int_{-\pi}^\pi dx\int_{-\pi}^\pi dy
  \int_{-\pi}^\pi dz
  \frac{1}{\omega-2t(\cos x+\cos y+\cos z)}.
  \label{simple-cubic-green-function}
\end{equation}

This function can be expressed by complete elliptic integral. The
complete elliptic integral of the first kind $K(k)$ as complex
function of the complex modulus $k$ is defined by
\begin{equation}
  K(k)=\int_0^{\frac{\pi}{2}}d\theta(1-k^2\sin^2\theta)^{-\frac{1}{2}},
\end{equation}
this function is an even function and $K(k^*)=K(k)^*$.

After the integration over $y$ and $z$, the integral
\eqref{simple-cubic-green-function} yields 
\begin{equation}
  G(\omega)=\frac{1}{2\pi^2t}\int_0^\pi kK(k)dx,
\end{equation}
where
\begin{equation}
  k=\frac{4t}{\omega-2t\cos x}.
\end{equation}

For simple cubic lattice, $\Re G$ is an odd function of $\omega$ and
$\Im G$ is an even function:
\begin{equation}
  \Re\;G(\omega)=-\Re\;G(\omega),\quad \Im\;G(\omega)=\Im\;G(\omega),
\end{equation}
hence we have only to consider the range $0\le\omega\le6t$ in the
following. The Green's function can calculated numerically, when $0<\omega<2t$,
\begin{equation}
  \begin{array}{l}
    \displaystyle\Re\;G(\omega)=-\frac{1}{2\pi^2t}\int_0^{\cos^{-1}(\omega/2t)}
    dxK\left(\frac{1}{|k|}\right)+\frac{1}{2\pi^2t}
    \int_{\cos^{-1}(\omega/2t)}^{\pi}K\left(\frac{1}{k}\right),\\\vbox
    to 20pt{} \displaystyle\Im\;G(\omega)=\frac{1}{\pi^2}\int_0^\pi
    dxK \left(\frac{\sqrt{k^2-1}}{k}\right),\\
  \end{array}
\end{equation}
when $2t\le\omega<6t$,
\begin{equation}
  \begin{array}{l}
    \displaystyle\Re\;G(\omega)=\frac{1}{2\pi^2t}
    \int_0^{\cos^{-1}[(\omega-4t)/2t]}dxK\left(\frac{1}{k}\right)
    +\frac{1}{2\pi^2t}\int_{\cos^{-1}[(\omega-4t)/2t]}^\pi dxK(k),\\\vbox to 20pt{}
    \displaystyle\Im\;G(\omega)=\frac{1}{2\pi^2t}
    \int_0^{\cos^{-1}[(\omega-4t)/2t]}dx K\left(\frac{\sqrt{k^2-1}}{k}\right).\\
  \end{array}
\end{equation}

\chapter{Coherent States}
Coherent states is defined as the eigenstates of annihilation operator:
\begin{equation}
  a_\alpha\ket{\phi}=\phi_\alpha\ket{\phi}.
\end{equation}

\section{Boson Coherent States}
Boson coherent states:
\begin{equation}
  \ket{\phi}=e^{\sum_\alpha\phi_\alpha a_\alpha^\dag}\ket{0},\quad
  \bra{\phi}=\bra{0}e^{\sum_\alpha\phi_\alpha^* a_\alpha}\quad,
\end{equation}
where $\phi_\alpha$ is complex number.

The overlap of two coherent states:
\begin{equation}
  \inprod{\phi}{\phi'}=e^{\sum_\alpha\phi_\alpha^*\phi'_\alpha}\quad.
\end{equation}

The overcompleteness in the Fock space:
\begin{equation}
  \int\left(\prod_\alpha\frac{d\phi_\alpha^*d\phi_\alpha}{2\pi i}\right)
  e^{-\sum\phi_\alpha^*\phi_\alpha}\ket{\phi}\bra{\phi}=1,
\end{equation}
where
\begin{equation}
  \frac{d\phi_\alpha^*d\phi_\alpha}{2\pi i}=
  \frac{d({\rm Re}\phi_\alpha)d({\rm Im}\phi_\alpha)}{\pi}\quad.
\end{equation}

The trace of an operator $A$ in Fock space can be written as
\begin{equation}
  {\rm Tr}A=
  \int\left(\prod_\alpha\frac{d\phi_\alpha^*d\phi_\alpha}{2\pi i}\right)
  e^{-\sum\phi_\alpha^*\phi_\alpha}\bra{\phi}A\ket{\phi}\quad.
\end{equation}

The average particle number of a coherent state is
\begin{equation}
  \bar{N}=\frac{\bra{\phi}N\ket{\phi}}{\inprod{\phi}{\phi}}
  =\frac{\bra{\phi}\sum_\alpha a_\alpha^\dag a_\alpha\ket{\phi}}{\inprod{\phi}{\phi}}
  =\sum_\alpha\phi_\alpha^*\phi_\alpha,
\end{equation}
and the variance is
\begin{equation}
  \sigma^2=\frac{\bra{\phi}N^2\ket{\phi}}{\inprod{\phi}{\phi}}-\bar{N}^2
  =\bar{N}\quad.
\end{equation}
%%-----------------------------------------------------------------------------

\section{Grassmann Algebra}
The Grassmann numbers is defined to be anticommuting numbers:
\begin{equation}
  \xi_\alpha\xi_\beta+\xi_\beta\xi_\alpha=0,\quad \xi_\alpha^2=0\quad.
\end{equation}

The conjugation of a Grassmann number is defined as
\begin{equation}
  (\xi_\alpha)^*=\xi_\alpha^*,\quad(\xi_\alpha^*)^*=\xi_\alpha\quad.
\end{equation}

If $\lambda$ is a complex number,
\begin{equation}
  (\lambda\xi_\alpha)^*=\lambda^*\xi_\alpha,
  \label{anticommuting}
\end{equation}
and for any product of Grassmann numbers:
\begin{equation}
  (\xi_1\cdots\xi_n)^*=\xi_n^*\xi_{n-1}^*\cdots\xi_1^*\quad,
\end{equation}
and for combinations of Grassmann variables and creation and
annihilation operators
\begin{equation}
  \xi a+a\xi=0,\quad(\xi a)^\dag=a^\dag\xi^*\quad.
\end{equation}

Because of property \eqref{anticommuting}, 
\begin{equation}
  f(\xi)=f_0+f_1\xi,\quad 
  A(\xi^*,\xi)=a_0+a_1\xi+\bar{a}_1\xi^*+a_{12}\xi^*\xi,
\end{equation}
in particular,
\begin{equation}
  e^{-\lambda\xi}=1-\lambda\xi\quad.
\end{equation}

A derivative can be defined for Grassmann variable function,
\begin{equation}
  \frac{\partial}{\partial\xi}(\xi^*\xi)=
  \frac{\partial}{\partial\xi}(-\xi\xi^*)=-\xi^*\quad.
\end{equation}

And a integral can be defined as
\begin{equation}
  \int d\xi\,1=0,\quad\int d\xi\,\xi=1,
  \quad\int d\xi^*\,1=0,\quad\int d\xi^*\,\xi^*=1,
\end{equation}
to remember,
\begin{equation}
  \int d\xi=\frac{\partial}{\partial\xi},\quad
  \int d\xi^*=\frac{\partial}{\partial\xi^*}\quad.
\end{equation}
%%--------------------------------------------------------------------------


\section{Fermion Coherent States}
Fermion Coherent States is defined as
\begin{equation}
  \ket{\xi}=e^{-\sum_\alpha\xi_\alpha a_\alpha^\dag}\ket{0}
  =\prod_\alpha(1-\xi_\alpha a_\alpha^\dag)\ket{0},
\end{equation}
we can verify that $a_\alpha\ket{\xi}=\xi_\alpha\ket{\xi}$ by using
\begin{equation}
  \xi_\alpha\ket{0}=\xi_\alpha(1-\xi_\alpha a^\dag)\ket{0}\quad.
\end{equation}

Similarly, the adjoint of the coherent states is
\begin{equation}
  \bra{\xi}=\bra{0}e^{-\sum_\alpha a_\alpha\xi_\alpha^*}
  =\bra{0}e^{\sum_\alpha\xi_\alpha^*a_\alpha}\quad.
\end{equation}

The overlap of two coherent states is 
\begin{equation}
  \inprod{\xi}{\xi'}=\prod_\alpha(1+\xi_\alpha^*\xi'_\alpha)
  =e^{\sum_\alpha\xi_\alpha^*\xi'_\alpha}\quad.
\end{equation}

The closure relation can be written as
\begin{equation}
  \int\left(\prod_\alpha d\xi_\alpha^*d\xi_\alpha\right)
  e^{-\sum_\alpha\xi_\alpha^*\xi_\alpha}\ket{\xi}\bra{\xi}=1\quad.
\end{equation}

The trace of an operator $A$ in Fock space can be written as
\begin{equation}
  {\rm Tr}A=\int\left(\prod_\alpha d\xi_\alpha^*d\xi_\alpha\right)
  e^{-\sum_\alpha\xi_\alpha^*\xi_\alpha}\bra{-\xi}A\ket{\xi},
\end{equation}
note the anti periodic condition here.
%%------------------------------------------------------------------------------


\chapter{Linear Response}
\section{Perturbations Depending on Time}
We now seek the solution of the perturbed equation
\begin{equation}
  i\hbar\frac{\partial\Psi(t)}{\partial t}=[H_0+V(t)]\Psi(t),
\end{equation}
in the form of a sum
\begin{equation}
  \Psi(t)=\sum_ka_k(t)\psi_k(t),
\end{equation}
where the expansion coefficients $a_k(t)$ are functions of time,
and $\psi_k(t)$ are unperturbed stationary wave functions:
\begin{equation}
  i\hbar\frac{\partial\psi_k(t)}{\partial t}=H_0\psi_k(t)=E_k^{(0)}\psi_k(t).
\end{equation}
Therefore we obtain that
\begin{equation}
  i\hbar\sum_k\psi_k(t)\frac{da_k(t)}{dt}=\sum_ka_k(t)V(t)\psi_k(t),
\end{equation}
multiplying both sides of this equation on the left by $\psi_m(t)$ and
integrating then
\begin{equation}
  i\hbar\frac{da_m(t)}{dt}=\sum_kV_{mk}(t)a_k(t),
\end{equation}
where
\begin{equation}
  V_{mk}(t)=\bra{m}V\ket{k}e^{i\omega_{mk}t}=V_{mk}e^{i\omega_{mk}t},\quad
  \omega_{mk}=\frac{E_{m}^{(0)}-E_k^{(0)}}{\hbar}.
\end{equation}

Let the unperturbed wave function be $\psi_n(t)$, i.e. $a_n^{(0)}=1$
and $a_k^{(0)}=0$ for $k\ne n$. To find the first approximation, we
seek $a_k=a_k^{0}+a_k^{(1)}$, substituting $a_k=a_k(0)$ we find
\begin{equation}
  i\hbar \frac{da_k^{(1)}(t)}{dt}=V_{kn}(t),
\end{equation}
integrating it gives
\begin{equation}
  a_{kn}^{(0)}(t)=-\frac{i}{\hbar}\int V_{kn}e^{i\omega_{kn}t}dt.
\end{equation}
%%----------------------------------------------------------------------------

\section{Fermi Golden Rule}
Let the perturbation be
\begin{equation}
  V(t)=Ve^{-i\omega t},
\end{equation}
then
\begin{equation}
  a_{fi}=-\frac{i}{\hbar}\int_0^tV_{fi}(t)dt=
  -V_{fi}\frac{e^{i(\omega_{fi}-\omega)t}-1}{\hbar(\omega_{fi}-\omega)}.
\end{equation}
Therefore the squared modulus of $a_{fi}$ is
\begin{equation}
  |a_{fi}|^2=|V_{fi}|^2\frac{4\sin^2[\frac{1}{2}(\omega_{fi}-\omega)t]}
{\hbar^2(\omega_{fi}-\omega)^2},
\end{equation}
noticing that $\lim_{t\to\infty}\frac{\sin^a\alpha t}{\pi t\alpha^2}=\delta(\alpha)$ we have
\begin{equation}
  |a_{fi}|^2=\frac{2\pi}{\hbar}|V_{fi}|^2\delta(E_f-E_i-\hbar\omega)t.
\end{equation}
Thus the probability $dw_{fi}$ of the transition rate per unit time is
\begin{equation}
  dw_{fi}=\frac{2\pi}{\hbar}|V_{fi}|^2\delta(E_f-E_i-\hbar\omega).
\end{equation}

Another method to derive the above formula is that let 
\begin{equation}
  V(t)=Ve^{-i\omega t+\eta t},
\end{equation}
and integrating from $t=-\infty$ to $t=0$, then
\begin{equation}
  |a_{fi}|^2=\frac{1}{\hbar^2}|V_{fi}|^2\frac{e^{2\eta t}}
{(\omega_{fi}-\omega)^2+\eta^2}
\end{equation}
Then the transition rate is [note that
$\lim_{\eta\to0}\frac{\eta}{\pi(\alpha^2+\eta^2)}=\delta(\alpha)$]
\begin{equation}
  \frac{d}{dt}|a_{fi}|^2=\frac{2\pi}{\hbar}|V_{fi}|^2\delta(E_f-E_i-\hbar\omega).
\end{equation}
%%----------------------------------------------------------------------------


\section{The Generalized Susceptibility}
When there exists an external interaction, the perturbing operator can
be written as
\begin{equation}
  V=-xf(t),  
\end{equation}
where $x$ is the operator of the physical quantity concerned, and the
perturbing generalized force $f$ is a given function of time.

The quantum mean value $\bar{x}(t)$ is given by a formula of the type
\begin{equation}
  \bar{x}(t)=\int_0^\infty\alpha(\tau)f(t-\tau)d\tau,
\end{equation}
where $\alpha(\tau)$ being a function of time which depends on the
properties of the body.

Applying fourier transform on both sides of this formula
\begin{equation}
  \int_0^\infty\bar{x}(t)e^{i\omega t}dt=
  \int_0^\infty\alpha(\tau)f(t-\tau)e^{i\omega t}d\tau dt,
\end{equation}
we obtain that
\begin{equation}
  \bar{x}(\omega)=\alpha(\omega)f(\omega).
\end{equation}

If the function $f$ is purely monochromatic and is given by the real
expression
\begin{equation}
  f(t)=\frac{1}{2}(f_0e^{-i\omega t}+f_0^*e^{i\omega t}),
  \label{generalized-force}
\end{equation}
then we shall have 
\begin{equation}
  \bar{x}(t)=\frac{1}{2}[\alpha(\omega)f_0e^{-i\omega t}
    +\alpha(-\omega)f_0^*e^{i\omega t}]
  \label{generalized-x}
\end{equation}

The function $\alpha(\omega)$ has the similar properties as retarded
Green's function:
\begin{equation}
  \alpha(-\omega)=\alpha^*(\omega),
\end{equation}
i.e.,
\begin{equation}
  \Re\,\alpha(-\omega)=\Re\,\alpha(\omega),\quad
  \Im\,\alpha(-\omega)=-\Im\,\alpha(\omega).
\end{equation}
And the Kramers-Kronig relations:
\begin{equation}
  \Re\,\alpha(\omega)=-\frac{1}{\pi}P\int_{-\infty}^\infty
  \frac{\Im\,\alpha(\varepsilon)}{\omega-\varepsilon}d\varepsilon,\quad
  \Im\,\alpha(\omega)=\frac{1}{\pi}P\int_{-\infty}^\infty
  \frac{\Re\,\alpha(\varepsilon)}{\omega-\varepsilon}d\varepsilon.
\end{equation}

The energy change per unit time of the system is just
$dE/dt=\overline{\partial H/\partial t}$, since only the perturbation
$V$ in Hamiltonian depends on explicitly on time, we have
\begin{equation}
  \frac{dE}{dt}=-\bar{x}\frac{df}{dt}.
\end{equation}
Substituting $\bar{x}$ and $f$ from \eqref{generalized-force} and
\eqref{generalized-x} and averaging over time, the terms containing 
$e^{2i\omega t}$ vanish, and we obtain 
\begin{equation}
  Q=\frac{1}{4}i\omega(\alpha^*-\alpha)|f_0|^2=
  \frac{1}{2}\omega\Im\,\alpha(\omega)|f_0|^2,
\end{equation}
where $Q$ is the mean energy dissipated per unit time.

\section{The Fluctuation Dissipation Theorem}
Let us now assume that the system is at state $\ket{n}$ and is subject to a
periodic perturbation, described by the operator
\begin{equation}
  V=-xf=-\frac{1}{2}x(f_0e^{-i\omega t}+f_0^*e^{i\omega t}).
\end{equation}
Using Fermi Golden Rule, the transition rate from state $n$ to state
$m$ per unit time is given by
\begin{equation}
  w_{mn}=\frac{\pi|f_0|^2}{2\hbar^2}|x_{mn}|^2
  [\delta(\omega+\omega_{mn})+\delta(\omega+\omega_{nm})].
\end{equation}
The dissipation per unit time is given by
\begin{equation}
  Q=\sum_{m}w_{mn}\hbar\omega_{mn}=\frac{\pi}{2\hbar}|f_0|^2
  \sum_m|x_{mn}|^2[\delta(\omega+\omega_{mn})+\delta(\omega+\omega_{nm})]
  \omega_{mn},
\end{equation}
or, since the delta function zero except when their argument is zero,
\begin{equation}
  Q=\frac{\pi}{2\hbar}\omega|f_0|^2
  \sum_m|x_{mn}|^2[\delta(\omega+\omega_{nm})-\delta(\omega+\omega_{mn})],
\end{equation}
thus
\begin{equation}
  \Im\,\alpha(\omega)=\frac{\pi}{\hbar}\sum_m|x_{mn}|^2
  [\delta(\omega+\omega_{nm})-\delta(\omega+\omega_{mn})].
\end{equation}

Now define 
\begin{equation}
  (x^2)_\omega=\int_{-\infty}^\infty\frac{1}{2}
  \average{x(t)x(0)+x(0)x(t)}e^{i\omega t}dt,
\end{equation}
in canonical ensemble it is
\begin{equation}
  (x^2)_\omega=\pi\sum_{nm}\rho_n|x_{mn}|^2
  [\delta(\omega+\omega_{nm})+\delta(\omega+\omega_{mn})],
\end{equation}
where $\rho_n=e^{(F-E_n)/T}$, $E_n$ denotes the energy levels and $F$
is free energy. Since the summation is now over both $m$ and $n$,
these can be interchanged:
\begin{equation}
  \begin{array}{rcl}
    (x^2)_\omega&=&\displaystyle\pi\sum_{mn}(\rho_n+\rho_m)|x_{mn}|^2
    \delta(\omega+\omega_{nm})\\\vbox to 18pt{}
    &=&\displaystyle\pi\sum_{mn}\rho_n(1+e^{-\hbar\omega_{mn}/T})|x_{mn}|^2
    \delta(\omega+\omega_{nm})\\\vbox to 18pt{}
    &=&\displaystyle\pi(1+e^{-\hbar\omega/T})\sum_{mn}\rho_n|x_{mn}|^2
    \delta(\omega+\omega_{nm}).
  \end{array}
\end{equation}
Similarly, in canonical ensemble
\begin{equation}
  \Im\,\alpha(\omega)=\frac{\pi}{\hbar}(1-e^{-\hbar\omega/T})
  \sum_{mn}\rho_n|x_{nm}|^2\delta(\omega+\omega_{nm}),
\end{equation}
a comparison of these two expressions gives
\begin{equation}
  (x^2)_\omega=\hbar\Im\,\alpha(\omega)\coth\frac{\hbar\omega}{2T}.
\end{equation}
The mean square of the fluctuating quantity is given by the
integration
\begin{equation}
  \average{x^2}=\frac{\hbar}{\pi}\int_0^\infty\Im\,\alpha(\omega)
  \coth\frac{\hbar\omega}{2T}d\omega.
\end{equation}

\section{Kubo Greenwood Formula}
Now write the perturbing operator as
\begin{equation}
  V=-\int\vec{j}\cdot\vec{A}dx,
\end{equation}
let $\alpha(\omega)$ denotes the corresponding generalized
susceptibility then the mean energy dissipated per unit time and per
unit volume is
\begin{equation}
  Q=\frac{1}{2}\omega\Im\,\alpha(\omega)|\vec{A}|^2.
\end{equation}
However, this generalized susceptibility is not the conductivity, to
get the conductivity, recall that
\begin{equation}
  \vec{E}(t)=-\frac{\partial\vec{A}}{\partial t},
\end{equation}
therefore
\begin{equation}
  \vec{E}(\omega)=i\omega\vec{A},
\end{equation}
which means
\begin{equation}
  j(\omega)=\alpha(\omega)A(\omega)=\frac{\alpha(\omega)}{i\omega}E(\omega),
\end{equation}
or
\begin{equation}
  \sigma(\omega)=\frac{\alpha(\omega)}{i\omega}.
\end{equation}

Thus the dissipated term written in conductivity is just
\begin{equation}
  Q=\frac{1}{2}\Im\,\alpha(\omega)|A|^2=\frac{1}{2}\Re\,\sigma(\omega)\;|E|^2,
\end{equation}
and
\begin{equation}
  \Re\,\sigma=\frac{\Im\,\alpha}{\omega}=
  \frac{\pi}{\hbar\omega}\sum_{mn}(\rho_n-\rho_m)
  |j_{mn}|^2\delta(\omega+\omega_{nm}).
\end{equation}

Now there comes an assumption which is called ``independent particle
approximation'': we replace $\rho$ by single electron distribution
function $f$ and recall that $j=-ev$ then
\begin{equation}
  \Re\,\sigma=\frac{\hbar\pi e^2}{V}\sum_{mn}
  \frac{f_n-f_m}{\hbar\omega_{mn}}|v_{mn}|^2\delta(E_n+\hbar\omega-E_m),
\end{equation}
where $V$ is the volume which acts as normalized factor.
Notice that 
\begin{equation}
  \frac{f_n-f_m}{\hbar\omega_{mn}}\delta(E_n+\hbar\omega-E_m)=
  \int dE\frac{f(E)-f(E+\hbar\omega)}{\hbar\omega}
  \delta(E-E_n)\delta(E_n+\hbar\omega-E_m),
\end{equation}
then the formula of $\Re\,\sigma$ become
\begin{equation}
  \begin{array}{rcl}
    \Re\,\sigma(\omega)&=&\displaystyle\frac{\hbar\pi e^2}{V} \int
    dE\frac{f(E)-f(E+\hbar\omega)}{\hbar\omega}\sum_{nm}
    v_{nm}\delta(E_n+\hbar\omega-E_m)v_{mn}\delta(E-E_n)\\\vbox to
    20pt{} &=&\displaystyle\frac{\hbar e^2}{\pi V}\int dE
    \frac{f(E)-f(E+\hbar\omega)}{\hbar\omega}
    \Tr[v\Im\,G^R(E+\hbar\omega)v\Im\,G^R(E)].
  \end{array}
\end{equation}
For static conductivity, we have
\begin{equation}
  \lim_{\omega\to0}\Re\,\sigma(\omega)=
  \frac{\hbar e^2}{\pi V}\int dE
  \left(-\frac{\partial f}{\partial E}\right)
  \sum_k|\bra{k}v\ket{k}|^2|\Im\,G^R(E,k)|^2,
\end{equation}
or in three dimension
\begin{equation}
  \lim_{\omega\to0}\Re\,\sigma(\omega)=
  \frac{\hbar e^2}{\pi V}\int dE
  \left(-\frac{\partial f}{\partial E}\right)
  \frac{a^3}{(2\pi)^3}\int d^3k\; v^2(k)|\Im\,G^R(E,k)|^2.
\end{equation}

\section{Green Kubo Formula}
Let $\Psi^{(0)}_n$ be the wave function of the unperturbed system,
then applying equations of perturbations depending on time in first
approximation we have
\begin{equation}
  \Psi_n=\Psi^{(0)}_n+\sum_ma_{m}\Psi^{(0)}_m,
\end{equation}
where $a_m$ satisfy the equation
\begin{equation}
  i\hbar\frac{da_m}{dt}=V_{mn}e^{i\omega_{mn}t}=-\frac{1}{2}
  x_{mn}e^{i\omega_{mn}t}(f_0e^{-i\omega t}+f_0^*e^{i\omega t}).
\end{equation}
In solving this, we must assume that the perturbation is ``adiabatically''
applied until the time $t$ from $t=-\infty$, this means that we must put
$\omega\to\omega\mp i0$ in factors $e^{\pm i\omega t}$. Then
\begin{equation}
  a_m=\frac{1}{2\hbar}x_{mn}e^{i\omega_{mn}t}\left[
    \frac{f_0e^{-i\omega t}}{\omega_{mn}-\omega-i0}+
    \frac{f_0^*e^{i\omega t}}{\omega_{mn}+\omega+i0}\right].
\end{equation}

Accordingly,
\begin{equation}
  \begin{array}{rcl}
    \bar{x}&=&\displaystyle\int \Psi_n^*x\Psi_ndq\\\vbox to 20pt{}
    &=&\displaystyle\sum_m(a_mx_{nm}e^{i\omega_{nm}t}+
    a^*_{m}x_{mn}e^{i\omega_{mn}t})\\\vbox to 20pt{}
    &=&\displaystyle\frac{1}{2\hbar}\sum_mx_{mn}x_{nm}\left[
      \frac{1}{\omega_{mn}-\omega-i0}+\frac{1}{\omega_{mn}+\omega+i0}
      \right]f_0e^{-i\omega t}+\hbox{c.c},
  \end{array}
\end{equation}
it can be seen that
\begin{equation}
  \alpha(\omega)=\frac{1}{\hbar}\sum_m|x_{mn}|^2\left[
  \frac{1}{\omega_{mn}-\omega-i0}+\frac{1}{\omega_{mn}+\omega+i0}
      \right].
\end{equation}
This expression is the Fourier transform of the function
\begin{equation}
  \alpha(t)=\frac{i}{\hbar}\theta(t)\average{x(t)x(0)-x(0)x(t)}=-G^R(t),
\end{equation}
thus the we have the final result
\begin{equation}
  \alpha(\omega)=\frac{i}{\hbar}\int_0^\infty e^{i\omega t}
  \average{x(t)x(0)-x(0)x(t)}dt.
\end{equation}
\chapter{Small Polaron}
\section{Holstein Model}
The Hamiltonian of Holstein Model is
\begin{equation}
  H=-\sum_{\langle{i,j}\rangle}t_{ij}c_i^\dag c_j+g\sum_ic_i^\dag c_i(a_i+a_i^\dag)
  +\omega_0\sum_ia_i^\dag a_i,
  \label{Holstein}
\end{equation}
where $c_i^\dag$ ($c_i$) is creation (annihilation) operator for electron, and
$a_i^\dag$ ($a_i$) is creation (annihilation) operator for phonon.

The model possesses two independent control parameters:
\begin{equation}
  \lambda =g^2/\omega_0t,
\end{equation}
\begin{equation}
  \gamma=\omega_0/t.
\end{equation}
A third parameter can be conveniently introduced as a combination of
the above ones:
\begin{equation}
  \alpha=\lambda/\gamma=g/\omega_0.
\end{equation}

It is worth defining the following regimes and limits, which are relevant 
to the Holstein model:

\begin{enumerate}[(i)]
\item weak (strong) couplings $\lambda<1$ ($\lambda>1$);
\item small (large) phonon frequency $\gamma<1$ ($\gamma>1$);
\item multiphonon regime $\alpha^2>1$;
\item adiabatic limit $\omega_0=0$, finite $\lambda$.
\end{enumerate}
%%----------------------------------------------------------------------

\section{Weak Coupling Limit}
Consider zero density ($n=0$) and zero temperature ($T=0$) limits,
Green's function for a single electron can be defined as
\begin{equation}
  G_{ij}(t)=-i\bra{0}Tc_i(t)c_j^\dag(0)\ket{0},
\end{equation}
where $\ket{0}$ is the vacuum for phonons and electrons. There is only
one possible ordering ($t>0$), so the function is purely retarded.

Let $g\sum_ic_i^\dag c_i(a_i+a_i^\dag)$ acts as perturbation, we have
that
\begin{equation}
  G_{ij}(t)=-i\bra{0}Tc_i(t)c_j(0)^\dag S\ket{0},
\end{equation}
where
\begin{equation}
  S=Te^{-i\int dt [g\sum_ic_i^\dag c_i(a_i+a_i^\dag)]}\quad.
\end{equation}
The expansion of $S$ to second order of $g$ gives
\begin{equation}
  \begin{array}{rcl}
  G_{ij}(t)&=&-i\bra{0}Tc_i(t)c_j^\dag\ket{0}\\\vbox to 18pt{}
  & &-i\displaystyle\frac{g^2}{2}\int dt'dt''\sum_{kl}
  \bra{0}Tc_i(t)c_j^\dag c_k^\dag(t')c_k(t')c_l^\dag(t'')c_l(t'')
  [a_k(t')a_l^\dag(t'')+a_k^\dag(t')a_l(t'')]\ket{0},\\
  \end{array}
\end{equation}
apply Wick's theorem and recall that ($D$ is the Green's function for phonon)
\begin{equation}
  \begin{array}{l}
    \bra{0}a_k^\dag(t')a_l(t'')\ket{0}=0,\\\vbox to 18pt{}
    \bra{0}a_k(t')a_l^\dag(t'')\ket{0}=D_{kl}(t'-t'')=\delta_{kl}D_{kk}(t'-t''),\\
  \end{array}
\end{equation}
we can obtain that
\begin{equation}
  G_{ij}(t)=G^{(0)}_{ij}(t)+ig^2\sum_k\int dt'dt''G^{(0)}_{ik}(t-t')
  G^{(0)}_{kk}(t'-t'')D_{kk}(t'-t'')G_{kj}(t''),
\end{equation}
in frequency space, (note that $D_{kk}(t'-t'')=-ie^{-i\omega_0(t'-t'')}$)
\begin{equation}
  G_{ij}(\omega)=G^{(0)}_{ij}(\omega)+
  g^2\sum_{k}G^{(0)}_{ik}(\omega)G_{kk}^{(0)}(\omega-\omega_0)G^{0}_{kj}(\omega).
\end{equation}
Compare with the Dyson equation
\begin{equation}
  G_{ij}=G^{(0)}_{ij}+\sum_{kl}G^{(0)}_{ik}\Sigma_{kl}G_{lj}
  =G^{(0)}_{ij}+\sum_{kl}G^{(0)}_{ik}\Sigma_{kl}G^{(0)}_{lj}+\cdots\quad,
\end{equation}
it is clear to see that second order perturbation gives a local
($k$-independent) self energy:
\begin{equation}
  \Sigma_2(\omega)=g^2G^{(0)}(\omega-\omega_0).
\end{equation}

The electron effective mass, in the case of a local self-energy, is
easily calculated via
\begin{equation}
  \frac{m^*}{m}=\left.\frac{d(\omega-\Re\Sigma(\omega))}{d\omega}\right|_{E_0}
  =1-\left.\frac{d\Re\Sigma(\omega)}{d\omega}\right|_{E_0},
\end{equation}
where $E_0$ is the ground-state energy.
%%----------------------------------------------------------------------------

\section{Atomic Limit (Zero Temperature)}
The atomic limit is defined as the zero hopping case ($t=0$).
In this case, Hamiltonian \eqref{Holstein} can be diagonalized by 
the unitary Lang-Firsov transformation
\begin{equation}
  U=e^{-S},\quad S=-\alpha\sum_{i}c_i^\dag c_i(a_i-a_i^\dag).
\end{equation}
With the aid of Baker-Campbell-Hausdorff formula we have
\begin{equation}
  \begin{array}{l}
    \bar{c}_i=e^Sc_ie^{-S}=c_iX_i,\quad X_i=e^{\alpha(a_i-a_i^\dag)};\\\vbox to 18pt{}
    \bar{c}_i^\dag=e^Sc_i^\dag e^{-S}=c_i^\dag X_i^\dag, \quad
    X_i^\dag=e^{\alpha(a_i^\dag-a_i)};\\\vbox to 18pt{}
    \bar{a}_i=e^Sa_ie^{-S}=a_i-\alpha c_i^\dag c_i;\\\vbox to 18pt{}
    \bar{a}_i^\dag=e^Sa_i^\dag e^{-S}=a_i^\dag-\alpha c_i^\dag c_i;\\\vbox to 18pt{}
    \bar{H}=e^SHe^{-S}=-\frac{g^2}{\omega_0}\sum_ic_i^\dag c_i+
    \omega_0\sum_ia_i^\dag a_i.\\
  \end{array}
\end{equation}
After the transformation, we can see that the ground energy is
$\varepsilon_p=-g^2/\omega_0$, the excited state energy is 
$\varepsilon_p+n\omega_0$.

The static electron-displacement correlation function is defined as
$C_0=\langle n_i(a_i+a_i^\dag)\rangle$, apply Lang-Firsov
transformation it reads
\begin{equation}
  C_0=\langle n_i(a_i+a_i^\dag)\rangle-2\alpha\langle n_i\rangle
  =-2\alpha\langle n_i\rangle,
\end{equation}
at the ground state $n_i=1$, thus $C_0=-2\alpha$. Meanwhile, 
\begin{equation}
  \langle e^Sa^\dag ae^{-S}\rangle=\langle a^\dag a\rangle+
  \alpha^2\langle c^\dag c\rangle=\alpha^2.
\end{equation}

The electron Green's function can also be calculated after the
Lang-Firsov transformation\footnote{Mahan's Many-Particle Physics, page 221}:
\begin{equation}
  \begin{array}{rcl}
  G(t)&=&-i\bra{0}c(t)c^\dag\ket{0}\\\vbox to 18pt{}
  &=&-i\bra{0}cXe^{-i\bar{H}t}c^\dag X^\dag\ket{0}\\\vbox to 18pt{}
  &=&-i\displaystyle\sum_{mn}\bra{0}cX\ket{m}\bra{m}e^{-i\bar{H}t}\ket{n}\bra{n}
  c^\dag X^\dag\ket{0},\\
  \end{array}
\end{equation}
where $\ket{m}$ is the phonon state corresponding to $m$ phonons.

Using the Feynman result ($e^{A+B}=e^Ae^Be^{-\frac{1}{2}[A,B]}$), we have that
\begin{equation}
  X^\dag=e^{-\alpha^2/2}e^{\alpha a^\dag}e^{-\alpha a},\quad
  X=e^{-\alpha^2/2}e^{-\alpha a^\dag}e^{\alpha a},
\end{equation}
accordingly,
\begin{equation}
  \begin{array}{l}
  \displaystyle\bra{m}X^\dag\ket{0}=e^{-\alpha^2/2}\bra{m}e^{\alpha a^\dag}\ket{0}
  =e^{-\alpha^2/2}\sum_n\bra{m}\frac{\alpha^n}{\sqrt{n!}}\ket{n}
  =e^{-\alpha^2/2}\frac{\alpha^m}{\sqrt{m!}},\\\vbox to 18pt{}
  \displaystyle\bra{0}X\ket{m}=e^{-\alpha^2/2}\frac{\alpha^m}{\sqrt{m!}}.
  \end{array}
\end{equation}
Finally the electron Green's function is
\begin{equation}
  G(\omega)=\sum_{n}^\infty \frac{\alpha^{2n}e^{-\alpha^2}}{n!}
  \frac{1}{\omega-n\omega_0-\varepsilon_p}.
\end{equation}

Let us now consider the action of the hopping. After the Lang-Firsov
transformation, the hopping term becomes
\begin{equation}
  t_{ij}c_i^\dag c_j\quad\to\quad t_{ij}X_i^\dag X_jc_i^\dag c_j,
\end{equation}
consider Holstein approximation, which neglect phonon emission and
absorption during the hopping process, we have
\begin{equation}
  t_{ij}\bra{0}X_i^\dag X_j\ket{0}=t_{ij}\bra{0}X^\dag\ket{0}\bra{0}X\ket{0}
  =t_{ij}e^{-\alpha^2}.
\end{equation}
%%---------------------------------------------------------------------------

\section{Atomic Limit (Finite Temperature)}
The Lang-Firsov transformation is the same as zero temperature
case. Here we need to calculate $\bra{n}X^\dag\ket{n}$. We have that
\begin{equation}
  \begin{array}{rcl}
  e^{-\alpha a}\ket{n}&=&\displaystyle\sum_{m=0}^\infty
  \frac{(-\alpha)^m}{m!}a^m\ket{n}\\\vbox to 20pt{}
  &=&\displaystyle\sum_{m=0}^n\frac{(-\alpha)^m}{m!}\left[
  \frac{n!}{(n-m)!}\right]^{\frac{1}{2}}\ket{n-m},
  \end{array}
\end{equation}
and
\begin{equation}
  \bra{n}e^{\alpha a^\dag}=\sum_{m=0}^n\frac{\alpha^m}{m!}\left[
  \frac{n!}{(n-m)!}\right]^{\frac{1}{2}}\bra{n-m},
\end{equation}
therefore
\begin{equation}
  \bra{n}e^{\alpha a^\dag}e^{-\alpha a}\ket{n}=
  \sum_{m=0}^n\frac{(-\alpha^2)^m}{m!}\frac{n!}{m!(n-m)!}=L_n(\alpha^2),
\end{equation}
where $L_n(x)$ is Laguerre polynomial. Thus
\begin{equation}
  \bra{n}X^\dag\ket{n}=\bra{n}X\ket{n}=e^{-\alpha^2/2}L_n(\alpha^2).
\end{equation}

At finite temperature, the assumption is that we only average on
phonon according to temperature. (``cold'' electron in a thermalized
phonon bath). So at finite temperature the effective hopping amplitude is
\begin{equation}
  \begin{array}{rcl}
    &&\displaystyle t_{ij}(1-e^{-\beta\omega_0})^2\sum_{mn}e^{-\beta m\omega_0}
    \bra{m}X_i^\dag\ket{m}e^{-\beta n\omega_0}\bra{n}X_j\ket{n}
    \\\vbox to 20pt{}
    &=&\displaystyle t_{ij}e^{-\alpha^2}\left[(1-e^{-\beta\omega_0})
      \sum_{n=0}^\infty e^{-n\beta\omega_0}L_n(\alpha^2)\right]^2.\\
  \end{array}
\end{equation}
Recall that the generating function of Laguerre polynomials:
\begin{equation}
  \frac{e^{-xt/(1-t)}}{1-t}=\sum_{n=0}^\infty L_n(x)t^n,
\end{equation}
let $t=e^{-\beta\omega_0}$ and $x=\alpha^2$ we find that the effective
hopping amplitude is
\begin{equation}
t_{ij}e^{-S_T}, \quad  S_T=\alpha^2(1+2\average{n}_T).
\end{equation}

Now let us turn to electron Green's function, now defined as
\begin{equation}
  \begin{array}{rcl}
      G(t)&=&\displaystyle-i(1-e^{-\beta\omega_0})\sum_ne^{-\beta n\omega_0}
      \bra{n}c(t)c^\dag\ket{n}\\\vbox to 18pt{}
      &=&\displaystyle-i(1-e^{-\beta\omega_0})\sum_ne^{-\beta n\omega_0}
      \bra{0}c(t)X(t)c^\dag X^\dag\ket{0}\\\vbox to 18pt{}
      &=&\displaystyle-i(1-e^{-\beta\omega_0})\bra{0}c(t)c^\dag\ket{0}
      \sum_ne^{-\beta n\omega_0}\bra{n}X(t)X^\dag\ket{n}.\\\vbox to 18pt{}
  \end{array}
\end{equation}
According to Heisenberg equation of motion (with Hamiltonian
$\bar{H}$), we have that
\begin{equation}
  \begin{array}{l}
    c(t)=ce^{-i\varepsilon_pt},\quad c^\dag(t)=c^\dag e^{i\varepsilon_pt};\\
    a(t)=ae^{-i\omega_0t},\quad a^\dag(t)=a^\dag e^{i\omega_0t},\\
  \end{array}
\end{equation}
thus
\begin{equation}
  X(t)=e^{-\alpha^2}e^{-\alpha a^\dag e^{i\omega_0t}}e^{\alpha ae^{-i\omega_0t}}=
  e^{-\alpha^2}e^{-\alpha a^\dag(t)}e^{\alpha a(t)}
\end{equation}
and
\begin{equation}
  X(t)X^\dag=e^{-\alpha^2}e^{-\alpha a^\dag(t)}e^{\alpha a(t)}
  e^{\alpha a^\dag}e^{-\alpha a}.
\end{equation}
Now we write $e^{\alpha a(t)}e^{\alpha a^\dag}$ as\footnote{see Mahan's
  Many-Particle Physics, page 222}
\begin{equation}
  e^{\alpha a(t)}e^{\alpha a^\dag}=e^{\alpha a^\dag}[e^{-\alpha
      a^\dag}e^{\alpha a(t)}e^{\alpha a^\dag}],
\end{equation}
using Baker-Campbell-Hausdorff formula we get
\begin{equation}
  e^{-\alpha a^\dag}e^{\alpha a(t)}e^{\alpha a^\dag}=
  e^{\alpha^2e^{-i\omega_0t}}e^{\alpha a(t)}.
\end{equation}
Finally the electron Green's function is arranged into the desired form:
\begin{equation}
  G(t)=-i(1-e^{-\beta\omega_0})e^{-\alpha^2(1-e^{-i\omega_0t})}
  \bra{0}c(t)c^\dag\ket{0}\sum_ne^{-\beta n\omega_0}
  \bra{n}e^{\alpha a^\dag(1-e^{i\omega_0t})}e^{-\alpha a(1-e^{-i\omega_0t})}\ket{n},
\end{equation}
again using Laguerre polynomials we can prove that
\begin{equation}
  (1-e^{-\beta\omega_0})\sum_ne^{-\beta n\omega_0}
  \bra{n}e^{u^*a^\dag}e^{-u a}\ket{n}=e^{-|u|^2/(e^{\beta\omega_0}-1)},
\end{equation}
thus
\begin{equation}
  G(t)=-ie^{-i\varepsilon_pt}
    \exp\Bigl[-\alpha^2[(N+1)(1-e^{-i\omega_0t})+N(1-e^{i\omega_0t})]\Bigr],
\end{equation}
where
\begin{equation}
  N=\frac{1}{e^{\beta\omega_0}-1}.
\end{equation}
Recall the generating function of Bessel functions of complex argument,
\begin{equation}
  e^{z\cos\theta}=\sum_{n=-\infty}^\infty I_n(z)e^{in\theta},
\end{equation}
let [note $(N+1)/N=e^{\beta\omega_0},\sqrt{(N+1)/N}=e^{\beta\omega_0/2}$]
\begin{equation}
  z=2\alpha^2\sqrt{N(N+1)},\quad \theta=\omega_0(t+i\beta/2)
\end{equation}
then (note that $I_{n}=I_{-n}$)
\begin{equation}
  G(t)=-ie^{-(2N+1)\alpha^2}e^{-i\varepsilon_pt}\sum_{n=-\infty}^\infty
  e^{-in\omega_0t}e^{\beta n\omega_0/2}I_n\{2\alpha^2\sqrt{N(N+1)}\},
\end{equation}
in frequency space
\begin{equation}
  G(\omega)=e^{-(2N+1)\alpha^2}\sum_{n=-\infty}^{\infty}
  e^{\beta n\omega_0/2}I_n\{2\alpha^2\sqrt{N(N+1)}\}
  \frac{1}{\omega-n\omega_0-\varepsilon_p}.
\end{equation}

%%---------------------------------------------------------------------

\section{The Impurity Analogy for A Single Electron}
The Hamiltonian for impurity model is
\begin{equation}
  H_{\rm imp}=\sum_k\varepsilon_kc_k^\dag c_k+\sum_kV_k(c_k^\dag d+d^\dag c_k)
  +\omega_0a^\dag a+gd^\dag d(a+a^\dag),
\end{equation}
here $V_k$ and $E_k$ is related to $G_0$ by
\begin{equation}
  G_0^{-1}(\omega)=\omega-\int_{-\infty}^\infty d\varepsilon
  \frac{\Delta(\varepsilon)}{\omega-\varepsilon},
\end{equation}
where
\begin{equation}
  \Delta(\varepsilon)=\sum_kV_k^2\delta(\varepsilon-\varepsilon_k).
\end{equation}

Let us separate the Hamiltonian into two parts $H_0$ and $V$, where
\begin{equation}
  H_0=\sum_k\varepsilon_kc_k^\dag c_k+\sum_kV_k(c_k^\dag d+d^\dag c_k)
  +\omega_0a^\dag a,\quad V=gd^\dag d(a+a^\dag).
\end{equation}

%%----------------------------------------------------------------------
\subsection{The Zero Temperature Formalism}
The Green's function for one electron at zero temperature is
\begin{equation}
  G(t)=-i\theta(t)\bra{0}d(t)d^\dag\ket{0},
\end{equation}
after Fourier transformation:
\begin{equation}
  G(\omega)=\bra{0}d\frac{1}{\omega+i0-H}d^\dag\ket{0}.
\end{equation}
  
An operator identity holds:
\begin{equation}
  \frac{1}{\omega-H}=\frac{1}{\omega-H_0}+
  \frac{1}{\omega-H_0}V\frac{1}{\omega-H}.
\end{equation}

To proceed further one needs to introduce the generalized matrix
elements:
\begin{equation}
  G_{nm}=\bra{0}\frac{a^n}{\sqrt{n!}}d\frac{1}{\omega-H}
  d^\dag\frac{(a^\dag)^m}{\sqrt{m!}}\ket{0},
\end{equation}
now introduce a set of zero electron $p$-phonon states and a set of
one electron $p$-phonon states
\begin{equation}
  \ket{0,p}=\frac{(a^\dag)^p}{\sqrt{p!}}\ket{0},
  \quad\ket{1,p}=\frac{(a^\dag)^p}{\sqrt{p!}}d^\dag\ket{0},
\end{equation}
one can write
\begin{equation}
  \begin{array}{rcl}
    G_{nm}&=&\displaystyle\bra{0}\frac{a^n}{\sqrt{n!}}d
    \frac{1}{\omega-H_0}d^\dag\frac{(a^\dag)^m}{\sqrt{m!}}\ket{0}+
    \bra{0}\frac{a^n}{\sqrt{n!}}d\frac{1}{\omega-H_0}V
    \frac{1}{\omega-H}d^\dag\frac{(a^\dag)^m}{\sqrt{m!}}\ket{0}\\\vbox to 20pt{}
    &=&\displaystyle G^{(0)}_{nm}+
    g\sum_{p_1,p_2}\bra{0}\frac{a^n}{\sqrt{n!}}d\frac{1}{\omega-H_0}d^\dag
    \ket{0,p_1}\bra{0,p_1}d(a+a^\dag)\ket{1,p_2}\bra{0,p_2}d
    \frac{1}{\omega-H}d^\dag\frac{(a^\dag)^m}{\sqrt{m!}}\ket{0}\\\vbox to 20pt{}
    &=&\displaystyle G^{(0)}_{nm}+
    g\sum_{p_1,p_2}G^{(0)}_{n,p_1}X_{p_1,p_2}G_{p_2,m}\\\vbox to 20pt{}
    &=&\displaystyle G^{(0)}_{nn}\delta_{nm}+
    g\sum_{p}G^{(0)}_{nn}X_{np}G_{pm}\quad,
  \end{array}
  \label{Polaron-Zero-Temperature-Element}
\end{equation}
where $G^{(0)}_{nn}(\omega)=G^{(0)}_{00}(\omega-n\omega_0)$ is the diagonal
element of the free Green's function, $X_{np}$ are the phonon
displacement matrix elements:
\begin{equation}
  X_{np}=\sqrt{p+1}\delta_{n,p+1}+\sqrt{p}\delta_{n,p-1}.
\end{equation}

Equation \eqref{Polaron-Zero-Temperature-Element} can be solved in
matrix notation:
\begin{equation}
  G^{-1}=G_0^{-1}-gX,
\end{equation}
it is easy to that $G^{-1}$ is a tridiagonal matrix.

Now define $T_k$ as the determinant of $G^{-1}$ with first $k$ rows
and columns removed, using Cramer's rule we find that
\begin{equation}
  G_{00}=\frac{T_1}{T_0},
\end{equation}
and define $D_k$ as the determinant comprising the first $k+1$ rows and
columns of $G^{-1}$ and $D_{-1}=1, D_{-2}=0$, then
\begin{equation}
  \begin{array}{l}
    D_0=[G^{(0)}]^{-1}_{00},\\
    D_1=[G^{(0)}]^{-1}_{11}[G^{(0)}]^{-1}_{00}-g^2
    =[G^{(0)}]^{-1}_{11}D_0-g^2,\\
    D_2=\cdots=[G^{(0)}]^{-1}_{2}D_1-2g^2D_0,\\
  \end{array}
\end{equation}
and, for the general case, the recurrence relations
\begin{equation}
  D_{k}=[G^{(0)}]^{-1}_{k,k}D_{k-1}-kg^2D_{k-2}.
\end{equation}
What's more, we have that
\begin{equation}
  T_k=[G^{(0)}]^{-1}_{kk}T_{k+1}-(k+1)g^2T_{k+2},\quad{\rm or}\quad
  \frac{T_k}{T_{k+1}}=[G^{(0)}]^{-1}_{kk}-(k+1)g^2\frac{T_{k+2}}{T_{k+1}},
\end{equation}
therefore
\begin{equation}
  \frac{T_1}{T_0}=\frac{1}{[G^{(0)}]^{-1}_{00}-g^2\frac{T_2}{T_1}}=\cdots,
\end{equation}
or
\begin{equation}
  G(\omega)=\frac{1}{\displaystyle G_0^{-1}(\omega)-
    \frac{g^2}{\displaystyle G_0^{-1}(\omega-\omega_0)-
      \frac{2g^2}{\displaystyle G_0^{-1}(\omega-2\omega_0)-
        \frac{3g^2}{\displaystyle G_0^{-1}(\omega-3\omega_0)-\cdots.}}}}
\end{equation}
Now use Dyson equation $\Sigma=G_0^{-1}-G^{-1}$ and we shall get
\begin{equation}
  \Sigma(\omega)=\frac{g^2}{\displaystyle G_0^{-1}(\omega-\omega_0)-
      \frac{2g^2}{\displaystyle G_0^{-1}(\omega-2\omega_0)-
        \frac{3g^2}{\displaystyle G_0^{-1}(\omega-3\omega_0)-\cdots.}}}
\end{equation}
The self-energy can be defined recursively,
\begin{equation}
  \Sigma^{(p)}(\omega)=\frac{pg^2}{G_0^{-1}(\omega-p\omega_0)-
    \Sigma^{(p+1)}}\quad.
\end{equation}

%%-----------------------------------------------------------------------
\subsection{The Finite Temperature Formalism}
At finite temperature, the trace performed over free phonon states gives
\begin{equation}
  G(\omega)=(1-e^{\beta\omega_0})\sum_ne^{-\beta n\omega_0}G_{nn}(\omega).
\end{equation}
Now we need to calculate $G_{nn}(\omega)$, according to $G^{-1}G=I$ we
have such a recurrence relation (recall that $G^{-1}$ is a tridiagonal
matrix):
\begin{equation}
  G_{nn}=G^{(0)}_n+gG^{(0)}_n(\sqrt{n}G_{n-1,n}+\sqrt{n+1}G_{n+1,n}),
\end{equation}
which we seek to write in a form as
\begin{equation}
    G_{nn}=G^{(0)}_n+G^{(0)}_n(AG_{nn}+BG_{nn}).
\end{equation}

Again according to Cramer's rule, 
\begin{equation}
  G_{n-1,n}=\sqrt{n}g\frac{D_{n-2}T_{n+1}}{T_0},\quad
  G_{nn}=\frac{D_{n-1}T_{n+1}}{T_0},
\end{equation}
recall the recurrence relation for $D$:
\begin{equation}
  D_k=[G^{(0)}_k]^{-1}D_{k-1}-kg^2D_{k-2},
\end{equation}
or
\begin{equation}
  \frac{D_{k-1}}{D_k}=\frac{1}{\displaystyle[G^{(0)}_k]^{-1}-kg^2
    \frac{D_{k-2}}{D_{k-1}}}\quad.
\end{equation}
Therefore 
\begin{equation}
  G_{n-1,n}=\sqrt{n}g\frac{D_{n-2}}{D_{n-1}}\frac{D_{n-1}T_{n+1}}{T_0}=
  \sqrt{n}g\frac{D_{n-2}}{D_{n-1}}G_{nn},
\end{equation}
i.e.,
\begin{equation}
  A=ng^2\frac{D_{n-2}}{D_{n-1}}=
  \frac{ng^2}{\displaystyle
    [G^{(0)}_n(\omega+\omega_0)]^{-1}-\frac{(n-1)g^2}{\displaystyle
      [G^{(0)}_n(\omega+2\omega_0)]^{-1}-\frac{(n-2)g^2}{\displaystyle
        \ddots-\frac{g^2}{[G^{(0)}_n(\omega+n\omega_0)]^{-1}}}}}
\end{equation}

Similarly, 
\begin{equation}
  G_{n+1,n}=\sqrt{n+1}g\frac{D_{n-1}T_{n+2}}{T_0}
  =\sqrt{n+1}g\frac{T_{n+2}}{T_{n+1}}G_{nn},
\end{equation}
recall the recurrence relation for $T$:
\begin{equation}
  T_k=[G^{(0)}_k]^{-1}T_{k+1}-(k+1)g^2T_{k+2},
\end{equation}
or
\begin{equation}
  \frac{T_{k+1}}{T_k}=\frac{1}{\displaystyle
    [G^{(0)}_k]^{-1}-(k+1)g^2\frac{T_{k+2}}{T_{k+1}}}.
\end{equation}
Therefore 
\begin{equation}
  B=(n+1)g^2\frac{T_{n+2}}{T_{n+1}}=
  \frac{(n+1)g^2}{\displaystyle
    [G^{(0)}_n(\omega-\omega_0)]^{-1}-\frac{(n+2)g^2}{\displaystyle
      [G^{(0)}_n(\omega-2\omega_0)]^{-1}-\frac{(n+3)g^2}{\displaystyle
        [G^{(0)}_n(\omega-3\omega_0)]^{-1}-\cdots,}}}
\end{equation}
finally 
\begin{equation}
  G_{nn}=\frac{1}{[G^{(0)}_n]^{-1}-A-B}.
\end{equation}
%%------------------------------------------------------------------------

\subsection{Dynamical Mean Field}
If we want to apply dynamical mean field theory, then a self
consistent condition is needed. Basically it is (see the solution for
simple impurity model)
\begin{equation}
  G^{-1}(\omega)=\omega-\sum_{ij}t_{oi}t_{jo}G^{(o)}_{ij}(\omega),
  \label{DMFT-self-consistent}
\end{equation}
where $G^{(o)}_{ij}$ is the Green's function with one site removed.
For Bethe lattice, it is very simple, in this case it is restricted
$i=j$, and in limit of infinite connectivity
$G^{(o)}_{ii}=G_{ii}$. Therefore for Bethe lattice
\begin{equation}
  G^{-1}(\omega)=\omega-t^2G(\omega).
\end{equation}
For a general lattice, the relation between the cavity and full
Green's functions reads
\begin{equation}
  G^{(o)}_{ij}=G_{ij}-\frac{G_{io}G_{oj}}{G_{oo}}.
\end{equation}

Therefore equation \eqref{DMFT-self-consistent} become
\begin{equation}
  G^{-1}=\omega-\sum_{ij}t_{oi}t_{jo}G_{ij}+
  \frac{\left(\sum_iG_{oi}\right)^2}{G_{oo}},
\end{equation}
recall that 
\begin{equation}
  G(\omega,k)=\frac{1}{\omega-\varepsilon_k-\Sigma(\omega)},
\end{equation}
we have that
\begin{equation}
  G^{-1}=\omega-\int d\varepsilon
  \frac{\rho(\varepsilon)\varepsilon^2}{\zeta-\varepsilon}
  -\left.\left(\int d\varepsilon
  \frac{\rho(\varepsilon)\varepsilon}{\zeta-\varepsilon}\right)^2\right/
  \int d\varepsilon\frac{\rho(\varepsilon)}{\zeta-\varepsilon},
\end{equation}
where $\zeta=\omega-\Sigma(\omega)$. This can be simplified further
using the following relations:
\begin{equation}
  \int d\varepsilon\frac{\rho(\varepsilon)\varepsilon^2}{\zeta-\varepsilon}
  =\zeta\int d\varepsilon
  \frac{\rho(\varepsilon)\varepsilon}{\zeta-\varepsilon},\quad
  \int d\varepsilon\frac{\rho(\varepsilon)}{\zeta-\varepsilon}=
  -1+\zeta\int d\varepsilon\frac{\rho(\varepsilon)}{\zeta-\varepsilon}.
\end{equation}
We have used $t_{oo}=\sum_kt_k=\int\rho(\varepsilon)\varepsilon=0$, finally
\begin{equation}
  G_0^{-1}=\Sigma+G^{-1}.
\end{equation}
\chapter{Physical Constants}
\begin{itemize}
\item The speed of light in vaccum, $c=299,792,458\;{\rm
  m/s}\approx3\times10^8\;{\rm m/s}$.
\item Electric charge $e=-1.602\times10^{-19}\;{\rm C}$.
\item energy in SI unit, joule $\rm J=kg\cdot(m/s)^2=N\cdot m=C\cdot V$.
\item Plank constant $h=6.62607004\times10^{-34}\;\rm{J\cdot s}=
  4.135667662\times10^{-15}\;{\rm eV\cdot s}$.
\item reduced Plank constant $\hbar=1.0545718\times10^{-34}\;{\rm J\cdot s}
  =6.582119514\times10^{-16}\;{\rm eV\cdot s}$
\item Boltzmann constant $k_B=1.38064852\times10^{-23}\;{\rm J\cdot K^{-1}}
  =8.6173324\times10^{-5}\;{\rm eV\cdot K^{-1}}$.
\item Bohr magneton $\mu_B=9.27400968\times10^{-24}\;{\rm J\cdot T^{-1}}
  =5.7883818066\times10^{-5}\;{\rm eV\cdot T^{-1}}$.
\item Bohr radius $a_0=5.29\times10^{-11}\;\rm m$.
\item Electron mass $m_e=9.10938215\times10^{-31}\;{\rm kg}
  =8.18710438\times10^{-14}\;{\rm J/c^2}=0.51099891\;{\rm MeV/c^2}$
\end{itemize}
\end{document}
